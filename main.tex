% --- Reverse-RAG 专用面上项目 LaTeX 模版 ---
% 编译引擎建议:XeLaTeX
% 参考文献处理:BibTeX

\documentclass[12pt, a4paper]{article}

% --- 1. 中文支持 ---
\usepackage[UTF8]{ctex} 

% --- 2. 引用与文献格式 ---
\usepackage[numbers,sort&compress]{natbib}
\usepackage{doi} 
\usepackage{hyperref} 
\hypersetup{colorlinks=true, linkcolor=blue, citecolor=blue, urlcolor=blue}

% --- 3. 页面布局 ---
\usepackage[margin=2.5cm]{geometry}
\renewcommand{\baselinestretch}{1.5} % 1.5 倍行距

% --- 4. 常用宏包 ---
\usepackage{amsmath, amssymb}
\usepackage{booktabs}
\usepackage{graphicx}

% --- 5. 文档元数据 ---
\title{2026 年度国家自然科学基金面上项目:\\ \textbf{基于 WORF-SEQ 的单倍型捕获测序技术开发}}
\author{郭泽华 \quad 博士/助理研究员}
\date{\today}

\begin{document}

\maketitle

% ==================================================================
% 使用说明:
% 1. 在 Overleaf [Menu] -> [Compiler] 中选择 [XeLaTeX]
% 2. 在 Overleaf 项目中新建 [refs.bib]
% 3. 将本系统生成的 [Report] 中的 BibTeX 源码粘贴进 [refs.bib]
% 4. 将本系统生成的 [Output] 正文内容粘贴在下方 \section 之后
% ==================================================================

\section{立项依据与科学问题}

% --- [在此粘贴 Output.md 的内容] ---

% --- [内容结束] ---

\newpage
\addcontentsline{toc}{section}{参考文献}
% 使用 unsrtnat 样式以支持 BibTeX 中的 DOI 字段显示
\bibliographystyle{unsrtnat} 
\bibliography{refs}

\end{document}